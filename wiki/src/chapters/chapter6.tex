\chapter{Conclusion}
\label{chapter:chapter6}

\section{Future Work}
\label{sec:future-work}

The current WordPressBench application is very simple, and performs only the basic functionalities. There are a lot of additions and bug fixes that could be implemented to improve the application and some of them are described below.

The system was first designed with a graphic interface, but due to lack of time it was not implemented anymore. A graphic interface would definitely be very useful because it would ease the input provisioning. The interface must have input controls for setting the number of users and the user should be allowed to set this variable during the simulation time as well. The interface should also allow to set the read-only ratio using percentage of values between 0 and 1.

Another desirable improvement concerns the display as well and represents the display of measurements in real-time. It is not an easy task, because this implies a periodic transmission of log files between Workload Generator and Controller and permanent aggregation of the log files by the Controller. In order to create the impression of real-time simulation, these operations have to be performed very often.

The transmission of messages between Controller and Workload Generators is done through TCP connections. They have initialization overhead and require more connection time, and the messages are not sent very frequently. To improve this issue, the connection could be replaced with a UDP connection in order to exchange short and fast datagrams.

The transition matrix is not generated very realistically. The current matrix contains probabilities based on empirical user behavior. An good improvement which would add realism to the simulation is to have a matrix based on measurements on real user behavior. The probabilities to go to certain states should be measured and put in relation with the read-only ratio.

During the evaluation process we discovered that the login state takes a longer amount of time, leading to spikes in graphics and a response time increase in certain states. If this issue is solved, the results would be more precise and the overall simulation would take less time.

User creation is another step which takes a larger amount of time than expected, and if fixed would have the same results as described above. This problem causes a delayed start of the simulation which can be observed in graphs on OX axis, where the relative time has large values, especially at simulations with larger amounts of users.

Finally, we consider that our evaluation process was a very basic one, and more extensive testing is needed. Using multiple machines for the Workload Generators would be a very good way of testing the scalability, as well as the Controller's capacity to deal with a lot of commands and log files. Flash crowds were one of the system's purpose, but we did not have enough time to simulate them either, so this could be one of the future tasks.

\section{Conclusion}
\label{sec:system-architecture}

The need of proper benchmarks has increased in recent years, because the existent ones are far from bringing the desired results. They have many limitations, like the lack configurability, realism, and scalability.

We designed WordPressBench to surpass the issues of present benchmarks. WordPressBench is a Web-hosting benchmark for testing various Web-server platforms. We designed the system with realistic user simulation in mind, achieved by using a real WordPress website. We aimed to bring scalability by adopting a master-slave architecture which allows the simulation of large number of users. Flexibility was the third goal, achieved by allowing the user to have control over the simulation. The user can set the number of users in real-time, the read-only ratio, or the simulation time.

WordPressBench is based on WordPress platform and emulates users by generating user requests on the fly, using pre-defined transition matrices. The system contains several Workload Generators which send HTTP requests to the WordPress website placed on the system under test. The Controller is the central component, sending commands to the Workload Generators and aggregating the log files.

The Workload Generators chose their states randomly from the transition matrices. They generate their URLs using data retrieved from previous requests and send them along with POST methods and cookies. There are three types of transition matrices depending on the user type: anonymous, registered, or read-only. The Workload Generators measure the response time at a certain moment in time, relative to the start of the simulation, and write these time measurements into log files. The files are sent to the Controller after the simulation resumes.

We evaluated the system by varying the number of users, the read-only ratio and the simulation time. The average response time per second at each second in time were plotted. We observed that the delay in login time and user creation creates spikes in graphs and delays the simulation start. We leave these issues to be solved in the future, along with a few other improvements.

To improve the current system we suggest a graphic interface which would permit the user to control the simulation process. A great feature of the graphic interface would be the display of results in real-time, or at least at very short time increments. We also propose the replacement of TCP communication with UDP datagrams. We also propose generating the transition matrix more scientifically, using precise measurements.
