\chapter{Conclusion}
\label{chapter:chapter6}

\section{Future Work}
\label{sec:future-work}

The current WordPressBench application is very simple, and performs only the basic functionalities. There are a lot of additions and bug fixes that could be implemented to improve the application and some of them are described below.

The sistem was first designed with a graphic interface, but due to lack of time it was not implemented anymore. A graphic interface would definatelyt be very useful because it would ease the input provisioning. The interface should have input controls for setting the number of users. The user should be allowed to set this variable during the simulation time as well. The interface should also allow to set the read-only ratio in percentage or value between 0 and 1.

Another desirable improvement concerns the display as well and represents the display of measurements in real-time. It is not an easy task, because this implies a periodic transmission of log files between Workload Generator and Controller and permanent aggregation of the logfiles by the Controller. In order to create the impression of real-time simulation, these operations have to be performed very often.

The transmission of messages between Controller and Workload Generatoors is done through TCP connections. They have initialization overhead and require more connection time, and the messages are not sent very frequently. To improve this issue, the connection could be replaced with a UDP connection in order to excange short and fast datagrams.

The transition matrix is not generated very realistically. The current matrix contains probabilities based on empirical user behavior. An good improvement which would add realism to the simulation is to have a matrix based on measurements on real user behavior. The probabilities to go to certain states should be measured and put in relation with the read-only ratio.

During the evaluation process we discovered that the login state takes a longer amount of time, leading to spikes in graphics and a response time increase in certain states.

User creation is another step which takes a larger amount of time than expected. This is the consequence of a delayed start of the simulation which can be observed in graphs on OX axis, where the relative time has large values, especially at simulations with larger amounts of users.

\section{Conclusion}
\label{sec:system-architecture}

!!!!!!!!!!!!!!!!!!!!!!!!!!!!!!!!!!!!!!!!!!!!!!!!!!!
