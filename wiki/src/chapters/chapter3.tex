\chapter{Challenges}
\label{chapter:chapter3}

Before designing the benchmark, we build up a list of requirements that needed to be satisfied by the benchmark. The first section describes these requirements. The functionalities and the results yield by the benchmark are described in the second section. In the design phase, we encountered a few impediments while trying to satisfy the requirements. These problems, along with a few generic steps for building a Web hosting benchmark are listed in the third section.

% WordPressBench was created with a few objectives in mind, the most important requirements being listed below in the first section, along with a brief explanation. The functionalities of the benchmark have emerged as soon as the requirements were settled, so the functionalities present in the graphic interface are described in the second section of this chapter. While building the benchmark we met a few challenges, and  the ones worth mentioning are present in the third section of this chapter, along with the chosen solutions.

\section{Requirements}
\label{sec:projectdescription}

WordPressBench aims at offering a \textbf{realistic user simulation}. This goal is achieved by generating variable traffic intensity represented by variable number of users at a certain time. In order to provide control over the simulation, the benchmark user can select the fluctuation range of the users number. The realism is also given by using WordPress, a real Web server platform used by tens of millions of users and running on millions of Web servers. It provides complex functionalities, advanced security and a MySQL database.

WordPressBench was designed with a \textbf{master-slave architecture} in mind. By following this architecture, scalability is provided. Therefore, 
it does not over-load the master when a large number of users are added and there is no risk of creating a bottleneck. Requests are sent from the slave machines to the Web servers running WordPress, which process the log files with the statistics data.

\textbf{Flexibility} is another major requirement of WordPressBench. This includes the possibility of setting the traffic intensity range through the graphic interface (the number of users). Another setting is defined by the read and write ratio, expressed in percentage, which indicate the predominant type of actions: reading or writing into the database. This benchmark, does not provide fixed configuration, or fixed scales. The results can be reproducible by using the same settings, although the workload data would be different, since it is generated randomly. Though the simulation results on the same set of Web servers, with the same settings, should remain the same.

\section{Functionalities}
\label{sec:functionalities}

WordPressBench does not offer a graphic interface, but generates log files ready to be send input to generate graphs. The log files contain the average response time and the corresponding relative time from the beginning of the simulation when it was measured. The time will be displayed on OX axis, and the average response time on the OY axis.
The user running the benchmark has the possibility of choosing the range of  traffic intensity, measured in number of users. The number of users will be fluctuating around the specified number. Another possible setting for the simulation is the read-write ratio, expressed in percentage. The read and write are the only two atomic actions available, so their percentage is complementary to each other, summing 100\%.

\section{Challenges}
\label{sec:challenges}

This section will provide a briefly description of the challenges we encountered during the design and implementation stages of WordPressBench. We will start with the explanations of some of the design decisions we made.

Benchmarks for stressing Web hosting systems are defined by large HTTP requests toward the resources of a certain website. The first step in designing our benchmark was to find a reliable website platform, able to support a large amount of users at once. We needed a real web platform, already developed and which allows distributed support. We considered that it is not the purpose of the project to build a new platform from scratch, which would take a considerable amount of time. We decided to choose among the open-source solutions, already available. WordPress platform came out immediately, because it is already used by millions of users and it is constantly updated. It is easy to install, highly extensible and it has been proved to be very reliable.

The second major design decision regarding WordPressBench was to define the user behavior. Our goal is to simulate the HTTP request as realistic as possible. Firstly, the requests needed to be generated from different machines. Since there is almost impossible to provide a machine for each request, we decided to have at least a group of requests generated from the same machine. This decision determined building a distributed set of workload generators. Besides distributing the user requests, user behavior would be simulated better if there is no constant number of users. Therefore, the benchmark user has the power of modifying the number of users at any moment in time.

The third issue that we faced was finding a way to implement read-write ratio of operations. Since for every write there must be at least one read to get the information first, we decided to have a read-only and read-write ratio. We first decided to have two types of websites, one for read-only working as a regular website, and one for blogging. When implementing, we realized that for the regular website we needed to generate content, and this had to be proportional with the number of requests. We finally decided to have a single blogging website for both read-only and read-write users whose behavior is determined by different transition matrices.

Regarding the development details of the workload generators, we faced difficulties when trying to login as a registered user. The login was performed using a user-name and password previously created by an administrator user, and it allows extensive actions, like adding new pages, new blog-posts, or comment as a registered user. The problem was that not only the cookies needed to be sent back, but also certain HTTP POST methods needed to be used. The solution was to analyze the login WordPress source code and determine which data was expected. 

We encountered a few design and implementation problems when building the real-time system. Our goal was to create an application which allows the benchmark user to modify the traffic intensity and view the statistics during the simulation. To do this, a large number of TCP requests are exchanged permanently between the Controller and all the Workload Generators. We used multi-threaded servers, to avoid blocking operations and a custom-made communication protocol between the components.
